\begin{table}[hp]
	\centering
	\caption{Fonctionnalités du serveur}
	\begin{tabular}{lP{2.5cm}P{2.5cm}P{2.5cm}P{2.5cm}l}
	\hline
	\bf Code & \bf Fonctionnalité & \bf Objectif & \bf Description & \bf Contraintes & \bf Priorité \\
	\hline
	\hline
	4.1 &
	Autorisation des paiements &
	Associer une transaction au bon compte &
	Doit être capable de savoir si le compte a assez d’argent &
	État autorisé ou non &
	1 \\\\
	4.2 &
	Autorisation des recharges &
	Associer un dépôt d’argent  au bon compte &
	Doit être capable d’ajouter le montant déposé au bon compte &
	État autorisé ou non &
	1 \\\\
	4.3 &
	Persistance des transactions et des états de compte &
	Être en mesure d’avoir un état du compte à tout moment &
	Le système doit enregistrer toutes les transactions reliées à un compte &
	À vie -- Procédure de backup &
	1 \\\\
	4.4 &
	Interface Web pour les usagers &
	Faire le suivi de ses achats &
	L’utilisateur doit être en mesure de voir toutes les achats qu’il a fait &
	Affichage du prix, de la quantité et de l’endroit &
	2 \\\\
	4.5 &
	Interface Web pour les vendeurs &
	Faire le suivi de ses ventes &
	Le vendeur doit être en mesure de voir toutes les ventes qu’il a fait &
	Affichage du prix, de la quantité et de l’endroit (ou client) &
	3 \\\\
	4.6 &
	Protéger le réseau contre la falsification &
	Sécuriser les transactions entre les différents noeuds &
	Utiliser une connexion sécurisée pour éviter que quelqu’un vole des données ou de l’argent &
	Encryption des communications &
	4 \\
	\hline
	\end{tabular}
	\label{cahierSer}
\end{table}
