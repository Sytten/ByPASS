\section{Obligations de formation}
	\subsection{Type et pièces justificatives}
		\subsubsection{Participation à des cours}
		Il s’agit d’une activité qui est structurée à l’aide d’un plan de cours.
		
		\paragraph{Formats admissibles}
		\begin{itemize}
			\item Cours crédités
			\item Cours collégiaux crédités
			\item Cours non crédités
			\item Études personnelles en vue d'un examen de certification
		\end{itemize}
		
		\paragraph{Éléments obligatoires pour l'admissibilité}
		\begin{itemize}
			\item Plan de cours
			\item Relevé de notes ou attestation officielle de réussite ou attestation officielle de participation
		\end{itemize}
		
		Il peut également avoir des heures d’étude personnelle admissible. L’admissibilité est conditionnelle à la réussite de l’examen.

		\subsubsection{Participation à des ateliers}
		Réunion de personnes pour approfondir une matière dans un but d’apprentissage. L’activité doit être dirigée par un animateur.

		\paragraph{Formats admissibles}
		\begin{itemize}
			\item Rencontre ponctuelle
			\item Accompagnement collectif
			\item Mentorat
			\item Coachine
			\item Parrainage
		\end{itemize}

		\paragraph{Éléments obligatoires pour l’admissibilité}
		\begin{itemize}
			\item Programmation de l'événement ou plan de l’activité ou entente de mentorat ou ordre du jour de l’activité
			\item Attestation de participation
		\end{itemize}
		
		\subsubsection{Participation à des séminaires}
		Réunion à caractère scientifique animé par un chercheur ou un spécialiste qui vise à faire le point sur l’état de l’art d’un domaine.

		\paragraph{Éléments obligatoires pour l’admissibilité}
		\begin{itemize}
			\item Programmation de l'événement ou plan de l’activité ou entente de mentorat ou ordre du jour de l’activité
			\item Attestation de participation
		\end{itemize}		
		
		\subsubsection{Participation à des conférences}
		Exposé oral visant l’acquisition de connaissances par les participants. Elle doit contenir une période de questions.
		
		\paragraph{Formats admissibles}
		\begin{itemize}
			\item Présentation orale
			\item Visite industrielle
		\end{itemize}
		
		\paragraph{Éléments obligatoires pour l’admissibilité}
		\begin{itemize}
			\item Programmation de l'événement ou plan de l’activité ou entente de mentorat ou ordre du jour de l’activité
			\item Attestation de participation
		\end{itemize}		
		
		\subsubsection{Participation à des comités techniques}
		Regroupement de personnes qualifiées partageant une même préoccupation technique qui contribue à l’amélioration de leurs activités professionnelles.

		\paragraph{Éléments obligatoires pour l’admissibilité}
		\begin{itemize}
			\item Documents présentant : l’objectif du comité, le plan de travail, le calendrier des rencontres et le nom des participants
			\item Comptes rendus des réunions signés par un supérieur immédiat
		\end{itemize}

		
		\subsubsection[Présentation d'un(e) cours/conférence, animation d'atelier/séminaire]{Présentation d'un cours ou d'une conférence,  animation d'un atelier ou séminaire}
		Le travail de recherche et d’analyse nécessaire à la préparation d’une telle activité visant à favoriser la formation continue doit permettre d’acquérir de nouvelles connaissances.

		\paragraph{Formats admissibles}
		\begin{itemize}
			\item Présentation d’un cours
			\item Présentation d’une conférence 
			\item Animation d’un atelier
			\item Animation d’un séminaire
		\end{itemize}

		\paragraph{Élément obligatoire pour l’admissibilité}
		\begin{itemize}
			\item Résumé de travail ou plan et contenu de l’activité
		\end{itemize}		
		
		\subsubsection{Rédaction d'articles et d'ouvrages spécialisés}
		Rédaction d’un document qui a pour but de contribuer au développement de l’exercice de la profession. Doit respecter les critères suivant:
		\begin{itemize}
			\item Doit développer des connaissances dans le domaine lié au métier de l’ingénieur
			\item Doit impérativement être publié pour être admissible
			\item Le texte doit être révisé par un comité de personnes compétentes
			\item Les instructions d’opérations, guides d’entretien, diffusions à usage interne ainsi que les rapports rédigés dans le cadre habituel du travail sont exclus.
		\end{itemize}
		
		\paragraph{Élément obligatoire pour l’admissibilité}
		\begin{itemize}
			\item Résumé de travail de recherche, d’analyse et de rédaction
			\item Copie de l’article ou de l’ouvrage avec les renseignement de publication
			\item Preuve de révision de l’article
		\end{itemize}
		
		\subsubsection{Participation à des activités d'auto-apprentissage ou à des projets de recherche}

		\paragraph{Formats admissibles}
		\begin{description}
			\item[Auto apprentissage] Activité dans le but d’améliorer ses compétences sans aide d’un formateur
			\item[Projet de recherche] Démarche visant au développement des connaissances de l’ingénieur
		\end{description}

		\paragraph{Élément obligatoire pour l’admissibilité}
		\begin{itemize}
			\item Documentation utilisée (références nécessaires)
			\item Résumé du contenu
			\item Résumé de l’objectif visé
			\item Dates de réalisation
		\end{itemize}

		Maximum de 5h par période de référence pour cette catégorie. 		
		
	\subsection{Fréquence et durée}
	L’ingénieur doit accumuler un minimum de 30~heures de formation continue admissibles par période de référence de deux (2)~ans. La première période de référence a débuté le 1er avril 2011. La formation continue est obligatoire à moins d’en être dispensé partiellement ou totalement.
	
	\subsection{Justification de formation}
	L’ingénieur doit fournir toutes les preuves justificatives admissibles selon le(s) type(s) de formation continue effectué(s) avant le 31 mai qui suit la fin de chaque période de référence en utilisant le formulaire prévu à cet effet sur le portail de l’Ordre des Ingénieurs du Québec.
