\section{Autres questions}
	\subsection{Question A}
	\emph{Est-ce que tous les ingénieurs doivent compléter leur formation périodique en même temps? Sinon, quel est le cycle pour cette obligation ?} \\
	
	Non, aucune restriction ne s’applique quand au moment où ces 30~heures de formation continue doivent être effectuées. Par contre, tous les membres inscrits au tableau doivent compléter leur formation continue dans la même période de référence de deux (2)~ans. 
	
	\subsection{Question B}
	\emph{Est-ce que des rapports techniques ou d’autres publications d’un ingénieur peuvent compter comme faisant partie de la formation continue ? Précisez les conditions.} \\
	
	Ces documents peuvent faire partie de la formation continue sous certaines conditions précises:
	\begin{itemize}
		\item Le but doit être de contribuer au développement des connaissances dans un domaine relié à la profession de l’ingénieur qui publie le document.
		\item Le texte doit être publié par un éditeur externe ou une association professionnelle et être disponible au public sous un format physique, électronique ou virtuel.
		\item Le texte doit avoir été révisé par un comité de personnes compétentes.
	\end{itemize}

	Certaines exclusions s’appliquent:
	\begin{itemize}
		\item Les textes diffusés à l’interne ou possédés entièrement par l’employeur de l’ingénieur
		\item Les documents techniques rédigés dans le cadre normal du travail de l’ingénieur
	\end{itemize}

	Si le texte est approuvé, l’ingénieur se voit créditer \SI{1}{h} de formation continue par tranche de 500~mots.

	\subsection{Question C}
	\emph{Quelle sera la conséquence si la formation continue d’un ingénieur n’est pas réalisée en temps ?}
	
	Un premier avis sera envoyé pour informer le membre en défaut qu’il a 90~jours pour fournir les preuves justificatives de ses heures de formation continue. Si ce dernier ne remplit pas son obligation, un deuxième avis est envoyé pour l’informer qu’il a un délai supplémentaire de 30~jours pour fournir les preuves justificatives, sous peine de quoi il sera radié du tableau de l’Ordre.

	\vfill

	{\centering%
	\pgfornament[width=5cm]{68}%
	\par}
	
	\vfill
