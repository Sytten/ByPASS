\section{Améliorations futures}
À la suite d’une deuxième itération, le produit devrait être assez complet pour tendre vers une première ronde de commercialisation. En effet, les principaux problèmes de sécurité auront été mitigés, l’interface utilisateur aura été améliorée et le matériel aura été testé sur une longue période.

Certaines étapes primordiales devront tout de même être réalisées. Premièrement, au niveau matériel, il est certain que les modèles des terminaux devront être recréés par ordinateur afin qu’ils puissent être produits en chaîne. Le bois sera probablement remplacé par du plastique et il sera essentiel d’engager un graphiste pour donner un meilleur \og look \fg{} au produit fini. Il sera aussi bien important d’assurer la sécurité de l’argent dans le terminal de recharge en ajoutant un mécanisme de protection. Finalement, un boîtier quelconque devra être conçu pour abriter le pont. 

Deuxièmement, au niveau électronique, il sera nécessaire de faire la conception des différents circuits imprimés du système afin qu’ils puissent être produits à la chaîne, testés facilement et assemblés rapidement dans les boîtiers. Cela évitera également les débranchements impromptus et les courts-circuits. Il sera aussi essentiel de concevoir un mécanisme de recharge des batteries. Notons que le produit devra obtenir les approbations nécessaires de la FCC\footnote{FCC : \emph{Federal Communications Commission}, une agence fédérale en charge de la régulation des télécommunications aux États-Unis.} et de l'ISDE\footnote{ISDE : \emph{Innovation, Sciences et Développement économique Canada}, un organisme fédéral favorisant l'accessibilité des entreprises canadiennes aux marchés mondiaux}.

Troisièmement, le logiciel serveur devra être distribué aux clients afin qu’ils puissent avoir leur propre instance locale. Il sera probablement nécessaire d’engager des techniciens pour aider à l’installation chez les clients et à la formation du personnel. 
