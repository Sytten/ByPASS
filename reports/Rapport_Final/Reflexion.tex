\section{Réflexion sur les corrections à apporter à la 2\ieme{} itération}
La première amélioration importante à réaliser pour la deuxième itération serait au niveau de la sécurité. En effet, pour le moment il n’y a aucune sécurité et c’est loin d’être idéal pour un système de paiement. Il faudrait commencer par chiffrer les communications Xbee pour éviter que des acteurs malveillants puissent espionner le réseau. Ensuite, il faudrait authentifier les n\oe{}uds sur le réseau afin de s’assurer qu’ils sont autorisés à effectuer des actions sur les comptes. Finalement, il faudrait aussi avoir un mécanisme d’authentification sur le site internet pour s’assurer que seul le propriétaire du compte puisse accéder à ses informations.

La deuxième amélioration qu'il serait important de réaliser est au niveau du code qui tourne sur les microcontrôleurs. Il faudrait essayer d'effectuer une gestion d’erreurs plus complète et avancée. Pour le moment, l’utilisateur a seulement une idée minimale des problèmes qui surviennent dans le système et le programme ne vérifie pas si les numéros de marchand ou de produits existent avant le paiement. Également, il faudrait que le système réessaye lorsque des erreurs de communication surviennent au lieu d'annuler la transaction. Finalement, il serait bien de faire un code un peu plus asynchrone sur les terminaux, qui sont très linéaires pour le moment, notamment pour permettre d’annuler ou modifier une transaction à n’importe moment dans le processus.

La troisième amélioration qui serait intéressante serait au niveau du matériel. Il faudrait d’abord incorporer le GPS tel qu’il était prévu dans le cahier des charges initial afin d’avoir une position des différents terminaux en tout temps. Ensuite, il faudrait tester de façon précise la durée de vie des terminaux sur batterie afin de pouvoir informer les clients et potentiellement les changer si nécessaire. Finalement, il faudrait essayer de réduire au minimum la consommation électrique en mettant en veille les différents appareils lorsque possible.

