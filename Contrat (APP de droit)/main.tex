\documentclass{article}

% Packages de base
\usepackage[french]{babel}
\usepackage[T1]{fontenc}
\usepackage[utf8]{inputenc}
\usepackage{microtype, lmodern}
\usepackage[hidelinks]{hyperref}
\usepackage{pdfpages}

% Mise en page
\usepackage[top=2.54cm, bottom=2.54cm, right=3.17cm, left=3.17cm]{geometry}
\usepackage{
	setspace,
	titlesec,
	etoolbox,
	multirow,
	caption,
	multicol,
	subfig,
	float,
	makecell,
	pdflscape,
	rotating,
	enumerate,
	enumitem}

\setlist[description]{leftmargin=0.5cm,labelindent=0.5cm}

\usepackage{xcolor, tikz}
\usetikzlibrary{babel} % Conciliation babel et tikz

\usepackage{array}
\newcolumntype{P}[1]{>{\raggedright\arraybackslash}p{#1}}

% Mathematiques
\usepackage{siunitx}
\sisetup{locale=FR}
\usepackage{
	amsmath,
	xfrac,
	commath,
	units,
	icomma,
	cancel}
\numberwithin{equation}{section}

% Lorem ipsum
\usepackage{chngcntr} % wtf... aucune idée
\counterwithin{figure}{section}
\usepackage{lipsum} % Faux-texte
\newcommand{\lipsumC}{\lipsum[\value{lipsumCTR}] \stepcounter{lipsumCTR}}
\newcounter{lipsumCTR} \setcounter{lipsumCTR}{1}

% Normes de l'universite
\titlespacing*{\section}{0pt}{18pt}{12pt} % IEEE standard
\titlespacing*{\subsection}{0pt}{12pt}{12pt} % IEEE standard
\titlespacing*{\subsubsection}{0pt}{12pt}{12pt} % IEEE standard

\newcommand{\ohs}{\onehalfspacing} % IEEE standard
\renewcommand{\theequation}{\thesection-\arabic{equation}} % IEEE standard
\renewcommand{\thefigure}{\thesection-\arabic{figure}} % IEEE standard
\renewcommand{\thetable}{\thesection-\arabic{table}} % IEEE standard
\addto\captionsfrench{\def\tablename{\sc{Tableau}}} % Renommer les tableaux
\patchcmd{\thebibliography}{\section*{\refname}}{}{}{} % Renommer la bibliographie

%: Commandes personnelles
\newcommand{\cip}[1]{#1}
\newcommand{\nomF}[1]{\textsc{#1}}
\newcommand{\code}[1]{\texttt{#1}}
\newcommand{\membre}[3]{#1 \textsc{#2} & \texttt{#3} \\}


\usepackage{comment}
\setlength\parindent{0pt}
\newcommand{\tofill}[1]{\emph{#1}}

\begin{document}

	%: PAGE TITRE
\begin{titlepage}
	\centering
		{\large\textsc{Université de Sherbrooke}} \\
		Faculté de génie \\
		Département de génie électrique et de génie informatique
        
        \vfill

		{\bfseries{\huge Contrat de travail}}
        
        \vfill

		Droit \\
		GEN501
        
        \vfill

		Présenté à \\
        Équipe professorale de la session S5
        
        \vfill

		Présenté par l'équipe 2 :\\ \smallskip%
        \begin{tabular}{r|l}
          \membre{Mathieu}{Dostie}{DOSM2902}
          \membre{Émile}{Fugulin}{FUGE2701}
          \membre{Philippe}{Girard}{GIRP2705}
          \membre{Damien}{Hulmann}{HULD1501}
          \membre{Julien}{Larochelle}{LARJ2526}
          \membre{Samir}{Lechekhab}{LECS2813}
          \membre{Donavan}{Martin}{MARD1206}
        \end{tabular}
        
        \vfill

		Sherbrooke -- \today %TODO Date de remise du rapport
        
\end{titlepage}

	
%	\pagenumbering{roman}
%	\tableofcontents
%	\clearpage
%    
%    \pagenumbering{arabic}

    \onehalfspacing
    
    {\centering \Large {\bfseries \uppercase{Contrat de Travail}} \\ \today \par} \vspace{2\baselineskip}
    
    % EMPLOYEUR
    \begin{minipage}[t]{0.45\textwidth}
    	\bfseries\uppercase{Entre :}
    \end{minipage}%
    \hfill%
    \begin{minipage}[t]{0.45\textwidth}
    	{\bfseries\uppercase{Nom de l'entreprise}}, personne morale dûment constituée en vertu de la Loi sur les compagnies partie 1A, ayant son siège social dans un paradis fiscal, représentée aux présentes par monsieur \tofill{Prénom Nom, poste du mandataire} dûment autorisé. \\ %TODO Entreprise/mandataire

    	{\raggedleft (ci-après appelée : \og l'Employeur \fg{}) \par}    	
    \end{minipage} \\ \vspace{\baselineskip}
    
    % EMPLOYE
    \begin{minipage}[t]{0.45\textwidth}
    	\bfseries\uppercase{Et :}
    \end{minipage}%
    \hfill%
    \begin{minipage}[t]{0.45\textwidth}
    	{\bfseries\uppercase{Monsieur} Prenom Nom}, domicilié et résidant au sous-sol de ses parents, Québec, Canada. \\ %TODO Employé
    	
    	{\raggedleft (ci-après appelé : \og l'Employé \fg{}) \par}
    \end{minipage} \\ \vspace{\baselineskip}
    
    % PARTIES
    \begin{minipage}[t]{0.45\textwidth}
    \end{minipage}%
    \hfill%
    \begin{minipage}[t]{0.45\textwidth}
    	(L'Employeur et l'Employé ci-après collectivement appelés \og les Parties \fg{})
    \end{minipage} \\ \vspace{\baselineskip}
    
    \hrule
    
    \section*{Préambule}
    
    \paragraph{ATTENDU QUE} l'Employeur désire engager l'Employé à titre de développeur;
    \paragraph{ATTENDU QUE} l'Employé désire accepter l'offre d'emploi de l'Employeur, après avoir pris connaissance des politiques et conditions d'emploi de celui-ci;
    \paragraph{ATTENDU QUE} l'Employé a reçu en mains propres les politiques et conditions en vigueur de l'Employeur et consultera périodiquement la version la plus à jour disponible sur l'intranet.
    \paragraph{ATTENDU QUE} les parties désirent confirmer leur entente par écrit; et
    \paragraph{ATTENDU QUE} les parties sont en mesure d'exercer tout droit requis pour la conclusion et l'exécution de l'entente comprise dans ledit contrat.
    
    \pagebreak
    
    {\bfseries\footnotesize\uppercase{Conséquemment à ce qui précède, les parties conviennent de ce qui suit :}}
    
    \section{Préambule}
    Le présent contrat inclut entièrement le préambule.
    
    \section{Objet}
    	\subsection{Poste occupé}
    	L'Employeur engage l'Employé à titre de développeur.
    	\subsection{Principales fonctions et responsabilités}
    	Non limitative et sujette à modification ultérieure sur préavis de l'Employeur, la description des principales fonctions et responsabilités de l'Employé est :
    	
    	\emph{Participer à la définition, au développement, tests, documentation, spécifications, et autres tâches relatives au développement de produits informatiques, ainsi que toute autre tâche qui lui serait attribuée par son supérieur immédiat.}
    	
    	\subsection{Supérieur immédiat}
    	L'Employé est sous la responsabilité immédiate de \tofill{Prénom Nom, Poste}, ou toute autre personne pouvant être désignée par l'Employeur.
    	
    	\subsection{Lieu de travail}
    	Le lieu de travail de l'Employé est situé au siège social de l'entreprise, ou à tout autre lieu requis pour l'exploitation efficace de l'entreprise par l'Employeur. Il n'est pas exclu que l'Employé doive parfois se travailler sur les lieux de travail des clients de l'entreprise.
    
    \section{Considération}
    	\subsection{Salaire}
    	En considération de l'exécution de son travail, l'Employé a droit à un salaire horaire de \SI{0}{\$}. Ledit salaire (diminué des déductions légales et contributives, s'il y a lieu) est payable à chaque deux semaines par chèque ou dépôt direct dans le compte de banque de l'Employé, au choix de l'Employeur.
    	
    	\subsection{Temps supplémentaire}
    	Tout temps supplémentaire doit être autorisé par l'Employeur préalablement à l'exécution par l'Employé. À la demande de l'Employé, l'Employeur peut substituer un congé payé d'une durée équivalente aux heures supplémentaires effectuées au paiement desdites heures supplémentaires, selon les dispositions prévues par la loi.
    	
    	\subsection{Vacances}
    	L'Employé a droit à \SI{4}{\%} de vacances payées annuellement. Les vacances ne sont ni monnayables ni cumulatives, et doivent être prises en tenant compte de l'ancienneté des autres employés et des besoins reliés à l'exploitation de l'entreprise par l'Employeur.
    	
    	\subsection{Congés fériés et payés}
    	L'Employé a droit aux jours de congé férié et payé établis par la loi. L'employé peut reprendre tout jour férié survenu pendant ses vacances immédiatement à la fin de celles-ci ou le différer.
    	
    	\subsection{Autres congés}
    	Les autres congés sont ceux établis par la loi ou ceux autorisés par le supérieur immédiat de l'Employé.
    
    \section{Dispositions particulières}
    \lipsumC
    \section{Dispositions générales}
    \lipsumC
    \section{Entrée en vigueur du contrat}
    \lipsumC
    \section{Durée du contrat}
    \lipsumC
    \section{Fin du contrat}
    \lipsumC
    \section{Reconnaissance des parties}
    \lipsumC
    
    

    
    
        
\end{document}
