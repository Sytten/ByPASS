\documentclass{article}

% Packages de base
\usepackage[french]{babel}
\usepackage[T1]{fontenc}
\usepackage[utf8]{inputenc}
\usepackage{microtype, lmodern}
\usepackage[hidelinks]{hyperref}
\usepackage{pdfpages}

% Mise en page
\usepackage[top=2.54cm, bottom=2.54cm, right=3.17cm, left=3.17cm]{geometry}
\usepackage{
	setspace,
	titlesec,
	etoolbox,
	multirow,
	caption,
	multicol,
	subfig,
	float,
	makecell,
	pdflscape,
	rotating,
	enumerate}

\usepackage{xcolor, tikz}
\usetikzlibrary{babel} % Conciliation babel et tikz

\usepackage{array}
\newcolumntype{P}[1]{>{\raggedright\arraybackslash}p{#1}}

% Mathematiques
\usepackage{siunitx}
\sisetup{locale=FR}
\usepackage{
	amsmath,
	xfrac,
	commath,
	units,
	icomma,
	cancel}
\numberwithin{equation}{section}

% Lorem ipsum
\usepackage{chngcntr} % wtf... aucune idée
\counterwithin{figure}{section}
\usepackage{lipsum} % Faux-texte
\newcommand{\lipsumC}{\lipsum[\value{lipsumCTR}] \stepcounter{lipsumCTR}}
\newcounter{lipsumCTR} \setcounter{lipsumCTR}{1}

% Normes de l'universite
\titlespacing*{\section}{0pt}{18pt}{12pt} % IEEE standard
\titlespacing*{\subsection}{0pt}{12pt}{12pt} % IEEE standard
\titlespacing*{\subsubsection}{0pt}{12pt}{12pt} % IEEE standard

\newcommand{\ohs}{\onehalfspacing} % IEEE standard
\renewcommand{\theequation}{\thesection-\arabic{equation}} % IEEE standard
\renewcommand{\thefigure}{\thesection-\arabic{figure}} % IEEE standard
\renewcommand{\thetable}{\thesection-\arabic{table}} % IEEE standard
\addto\captionsfrench{\def\tablename{\sc{Tableau}}} % Renommer les tableaux
\patchcmd{\thebibliography}{\section*{\refname}}{}{}{} % Renommer la bibliographie

%: Commandes personnelles
\newcommand{\cip}[1]{#1}
\newcommand{\nomF}[1]{\textsc{#1}}
\newcommand{\code}[1]{\texttt{#1}}
\newcommand{\membre}[3]{#1 \textsc{#2} & \texttt{#3} \\}


\usepackage{comment}
\usepackage{pgfornament}
\setlength\parindent{0pt}
\newcommand{\tofill}[1]{\emph{#1}}

\begin{document}

	%: PAGE TITRE
\begin{titlepage}
	\centering
		{\large\textsc{Université de Sherbrooke}} \\
		Faculté de génie \\
		Département de génie électrique et de génie informatique
        
        \vfill

		{\bfseries{\huge Rapport de projet} \\ {\Large Revue 2}}
        
        \vfill

		Conception d'un système embarqué et réseauté \\
		GIF500
        
        \vfill

		Présenté à \\
        M. Domingo Palao Muñoz
        
        \vfill

		Présenté par l'équipe 2 :\\ \smallskip%
        \begin{tabular}{r|l}
          \membre{Mathieu}{Dostie}{DOSM2902}
          \membre{Émile}{Fugulin}{FUGE2701}
          \membre{Philippe}{Girard}{GIRP2705}
          \membre{Damien}{Hulmann}{HULD1501}
          \membre{Julien}{Larochelle}{LARJ2526}
          \membre{Samir}{Lechekhab}{LECS2813}
          \membre{Donavan}{Martin}{MARD1206}
        \end{tabular}
        
        \vfill

		Sherbrooke -- \today %TODO Date de remise du rapport
        
\end{titlepage}

	
%	\pagenumbering{roman}
%	\tableofcontents
%	\clearpage
%    
%    \pagenumbering{arabic}

    \onehalfspacing
    
    {\centering \Large {\bfseries \uppercase{Contrat de Travail}} \\ \today \par} \vspace{2\baselineskip}
    
    % EMPLOYEUR
    \begin{minipage}[t]{0.45\textwidth}
    	\bfseries\uppercase{Entre :}
    \end{minipage}%
    \hfill\marginpar{\emph{(819) 943-0706}}%
    \begin{minipage}[t]{0.45\textwidth}
    	{\bfseries\uppercase{ByPass Inc.}}, personne morale dûment constituée en vertu de la Loi sur les compagnies par action partie 1, ayant son siège social au 2500 boulevard de l'Université, au local D4-2009, en la ville de Sherbrooke, province de Québec, J1K 2R1, représentée aux présentes par monsieur {Julien Larochelle, chargé de projet} dûment autorisé. \\ 

    	{\raggedleft (ci-après appelée : \og l'Employeur \fg{}) \par}    	
    \end{minipage} \\ \vspace{\baselineskip}
    
    % EMPLOYE
    \begin{minipage}[t]{0.45\textwidth}
    	\bfseries\uppercase{Et :}
    \end{minipage}%
    \hfill\marginpar{\emph{(819) 868-7866}}%
    \begin{minipage}[t]{0.45\textwidth}
    	{\bfseries\uppercase{Monsieur} Yvon Brulé}, domicilié et résidant au 1443 rue Gauvin, en la ville de Sherbrooke, province de Québec, J1K 2J2. \\
    	
    	{\raggedleft (ci-après appelé : \og l'Employé \fg{}) \par}
    \end{minipage} \\ \vspace{\baselineskip}
    
    % PARTIES
    \begin{minipage}[t]{0.45\textwidth}
    \end{minipage}%
    \hfill%
    \begin{minipage}[t]{0.45\textwidth}
    	(L'Employeur et l'Employé ci-après collectivement appelés \og les Parties \fg{})
    \end{minipage} \\ \vspace{\baselineskip}
    
    \hrule
    
    \section{Préambule et définitions}
    
    \paragraph{ATTENDU QUE} l'Employeur désire engager l'Employé à titre d'ingénieur logiciel;
    \paragraph{ATTENDU QUE} l'Employé désire accepter l'offre d'emploi de l'Employeur, après avoir pris connaissance des politiques et conditions d'emploi de celui-ci;
    \paragraph{ATTENDU QUE} l'Employé a reçu en mains propres les politiques et conditions en vigueur;
    \paragraph{ATTENDU QUE} les parties désirent confirmer leur entente par écrit; et
    \paragraph{ATTENDU QUE} les parties sont en mesure d'exercer avec compétence tout droit requis pour la mise en application de l'entente comprise dans ledit contrat.
    
    \vfill%
    %
    {\centering%
    \pgfornament[width=0.05\textwidth]{70}
    \pgfornament[width=0.05\textwidth]{70}
    \pgfornament[width=0.05\textwidth]{70}%
    \par}%
    %
    \vfill
    
    \pagebreak
    
%    {\bfseries\footnotesize\uppercase{Conséquemment à ce qui précède, les parties conviennent de ce qui suit :}}
    
%    \section{Préambule}
%    Le présent contrat inclut entièrement le préambule.
    
    \section{Durée et lieu du travail}

        \subsection{Durée du contrat}
	    Le présent contrat est à durée indéterminée, et commence le \today.
        	
    	\subsection{Lieu de travail}
    	Le lieu de travail de l'Employé se trouve au siège social de l'entreprise, au 2500 boulevard de l'Université, au local D4-2009, en la ville de Sherbrooke, province de Québec, J1K 2R1.
    	
   \section{Fonctions et responsabilités du travailleur}
   	L'Employeur engage l'Employé à titre d'ingénieur logiciel.

    Sans s'y limiter et sujette à modification ultérieure sur préavis de l'Employeur, les principales fonctions et responsabilités de l'Employé sont définies ainsi :
    	
    	\emph{Participer à la conception, au développement, aux rencontres avec les clients, à la documentation, aux tests d'intégration et de performance, ainsi qu'à toute autre tâche relative au développement logiciel et matériel des produits et services de l'Employeur.}
    
    \section{Rémunération}
    	\subsection{Salaire}
    	En contrepartie de l'exécution de son travail, l'Employé a droit à un salaire horaire de \SI{40}{\$}. Ce salaire est payable à chaque deux semaines par dépôt direct dans le compte de banque de l'Employé.
    	
    	\subsection{Assurance médicale}
    	L'Employé peut sur demande souscrire au régime d'assurances médicale et dentaire offert par l'Employeur.
    	
    	\subsection{Primes}
    	Aucune prime versée à l'Employé par l'Employeur n'est prévue dans le présent contrat.
    	
    	\subsection{Autres avantages sociaux}
    	L'Employeur s'engage à payer l'abonnement annuel de l'Employé à la salle d'entraînement du centre sportif du campus principal de l'Université de Sherbrooke, situé au siège social.
    	       	
%    	Tout matériel fourni demeure la propriété de l'Employeur.
    
    \section{Horaire de travail}
    	\subsection{Journées et semaines de travail}
    	L'Employé doit effectuer \SI{37,5}{h} de travail par semaine. Les heures normales de travail de l'Employé sont de 9h00 a.m. à 17h00 p.m., incluant trente minutes de pause non payée pour les repas. Il est convenu que l'Employé peut faire varier cette plage horaire de travail de 2~heures.
    	
    	La semaine de travail comprend 5~jours de 8~heures par semaine, du lundi au vendredi. Cette plage est cependant variable selon les fonctions de l'Employé ou les besoins de l'Employeur.
    	
    	\subsection{Périodes de repos}
    	Toute pause supplémentaire est aux frais de l'Employé.
    	
    	\subsection{Temps supplémentaire}
    	Tout temps supplémentaire doit être autorisé par l'Employeur préalablement à son exécution par l'Employé, et sera rémunéré en fonction des dispositions prévues par la loi.  	
    	
    \section{Vacances et congés}
    	\subsection{Vacances}
    	L'Employé a droit à \SI{4}{\%} de vacances payées annuellement. Les vacances ne sont ni monnayables ni cumulatives, et doivent être prises en tenant compte de l'ancienneté des autres employés et des besoins reliés à l'exploitation de l'entreprise par l'Employeur. Un préavis d'un mois est requis.    	
    	
    	\subsection{Congés fériés et payés}
    	L'Employé a droit aux jours de congé férié et payé établis par la loi :
    	
    	\begin{itemize}
    		\item le 1er janvier (jour de l’An);
			\item le Vendredi saint ou le lundi de Pâques, au choix de l’employeur;
			\item le lundi qui précède le 25 mai (Journée nationale des patriotes);
			\item le 24 juin (fête nationale);
			\item le 1er juillet. Si cette date tombe un dimanche : le 2 juillet (Fête du Canada);
			\item le 1er lundi de septembre (fête du Travail);
			\item le 2e lundi d’octobre (Action de grâces);
			\item le 25 décembre (jour de Noël).
    	\end{itemize}
    	
    	L'Employeur rémunère la période de fermeture des Fêtes, laquelle sera communiquée au début du mois de novembre de chaque année.
    	
    	 L'employé peut reprendre tout jour férié survenu pendant ses vacances immédiatement à la fin de celles-ci ou le différer.
    	
    	\subsection{Autres congés}
    	Les autres congés sont ceux autorisés par l'Employeur.
    	
    	\subsection{Congés familiaux et de maladie}
    	L'Employeur s'engage à respecter les congés familiaux et de maladie établis par la loi. Certaines exceptions s'appliquent pour les cas suivants :
    	
    	\begin{enumerate}
    		\item Dans le cas du décès d'un membre de la famille immédiate, l'Employé a droit à 3~jours de congé payés aux frais de l'Employeur et à 5~jours supplémentaires à ses frais.
    	
   			\item Dans le cas de la naissance d'un enfant, l'Employé qui a travaillé depuis au moins 60~jours de façon continue pour l'Employeur a droit à 5~jours de congé payés par enfant aux frais de l'Employeur et à 5~jours supplémentaires à ses frais. Autrement, l'absence est à ses frais.

   			\item Dans le cas de maladie, l'Employé a droit à 5~jours aux frais de l'Employeur par année.

    	\end{enumerate}
    	
    \section{Réserve de propriété intellectuelle}
    Dans le cadre de son travail, que ce soit ou non durant l'horaire de travail ou sur le lieu de travail, l'Employé pourrait être amené à créer, concevoir ou développer lui-même ou conjointement des procédés, logiciels, données, inventions, découvertes, diagrammes, plans, cahiers des charges, techniques, méthodes, algorithmes, dessins, protocoles, savoir-faire (brevets, marques de commerce, droits d'auteur et dessins industriels).
        L'Employé reconnaît et accepte que l'Employeur est propriétaire et titulaire de tout droit, titre et intérêt relatifs aux propriétés intellectuelles mentionnées ci-haut.
        L'Employé cède d'avance à l'Employeur tout droit qu'il pourrait prétendre posséder relativement aux propriétés intellectuelles mentionnées ci-haut.
    
    \section{Avis de cessation ou de démission}
    	\subsection{Cessation}
	    En cas de licenciement, l'Employeur s'engage à remettre un avis de cessation d'emploi à l'Employé en respectant les délais prévus par la loi, en vertu de l'ancienneté de l'Employé :
	    
	    \begin{itemize}
	    	\item 3 mois de travail continu : 1 semaine de préavis
	    	\item 1 an de travail continu : 2 semaines de préavis
	    	\item 5 ans de travail continu : 4 semaines de préavis
	    	\item 10 ans de travail continu : 8 semaines de préavis
	    \end{itemize} \bigskip
	    
	    Aucune compensation n'est prévue si ces délais sont respectés. Les exceptions suivantes s'appliquent quant au respect du délai, auxquelles s'y rattache un simple avis sans possibilité de compensation : force majeure, faillite de l'Employeur, décès de l'Employé, faute grave.
			    	
    	\subsection{Démission}
	    En cas de démission, l'Employé s'engage à remettre un avis de démission à l'Employeur dans les deux semaines précédant son départ.
    
    	\section{Clause de non-concurrence}
    	Afin de protéger les intérêts légitimes de l'Employeur, l'Employé s'engage  à ne pas faire concurrence à l'Employeur ni à participer à une entreprise faisant concurrence directe à l'Employeur.
    	
    		\subsection{Territoire}
    		L'engagement est valide pour le territoire suivant : un rayon de \SI{50}{km} à vol d'oiseau à partir du siège social de l'Employeur, au 2500 boulevard de l'Université, au local D4-2009, en la ville de Sherbrooke, province de Québec, J1K 2R1.
    		
    		\subsection{Durée}
    		L'engagement est valide durant la période de validité du contrat et durant les 6~mois suivant sa fin.
    		
    		\subsection{Domaine professionnel}
    		Le domaine visé par cet engagement est la conception de systèmes informatiques embarqués pour le paiement par carte RFID dans des réseaux d'organisations privées.
    		
    		\subsection{Défaut} Si l'Employé fait défaut par rapport au respect du présent engagement, il doit payer à l'Employeur tout dommage et intérêt que saurait lui être reconnue par l'Employeur. De plus, une somme pécunière minimale s'élevant à \SI{1000}{\$} par jour de violation du présent engagement sera réclamée à l'Employé.
    	
    \pagebreak
   	\section{Signature des parties}
   	En foi de quoi, les deux Parties ont signé aux lieux et dates indiqués ci-dessous par un représentant dûment autorisé de chacune des Parties : \\
    	
    \begin{minipage}[t]{0.45\textwidth}
    	{\bfseries Employeur} \\ \smallskip
    	
    	\underline{\hspace{\textwidth}} \\
    	Signature \\[0.5cm]
    	\underline{\hspace{\textwidth}} \\
    	Nom \\[0.5cm]
    	\underline{\hspace{\textwidth}} \\
    	Ville, Province \\[0.5cm]
    	\underline{\hspace{\textwidth}} \\
    	Date \\
    \end{minipage}%
    \hfill%
    \begin{minipage}[t]{0.45\textwidth}
    	{\bfseries Employé} \\ \smallskip
    	
    	\underline{\hspace{\textwidth}} \\
    	Signature \\[0.5cm]
    	\underline{\hspace{\textwidth}} \\
    	Nom \\[0.5cm]
    	\underline{\hspace{\textwidth}} \\
    	Ville, Province \\[0.5cm]
    	\underline{\hspace{\textwidth}} \\
    	Date \\
    \end{minipage}
    
    \vfill
    
    {\centering%
    \pgfornament[width=5cm]{70}%
    \par}
    
    \vfill
    
    
    

    
    
        
\end{document}
