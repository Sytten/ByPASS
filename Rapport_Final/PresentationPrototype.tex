\section{Présentation du prototype}
La première itération du prototype suit les requis établis en début de session. En effet, il comprend trois grandes parties : le terminal de recharge, le terminal de paiement et le pont/serveur.

Premièrement, à partir du terminal de recharge l’utilisateur peut savoir l’état de son compte et faire un dépôt d’argent. Le prototype accepte seulement les pièces de monnaie. Pour être en mesure d'effectuer une recharge, l’utilisateur doit présenter sa carte NFC. Le prototype est transportable et fonctionnel sur batterie, ce qui est utile pour l’avènement d'événements spontanés, comme les 5@8.

Deuxièmement, le terminal de paiement est le lien entre les consommateurs et les marchands. En effet, le marchand entre les articles achetés par le client pour obtenir le prix total des achats. Ensuite, le client doit présenter sa carte NFC pour effectuer le paiement. Une DEL et un écran transmettent les informations reliées aux transactions. Ce terminal fonctionne aussi sur batterie et il est facile d’installation, étant donnée sa forme compacte.

Troisièmement, le pont permet de transmettre les requêtes des terminaux au serveur pour effectuer les bonnes requêtes et obtenir les informations nécessaires. Un site internet permet aux utilisateurs de voir leur état de compte en temps réel. Pour les marchands, ils ont la possibilité de modifier leur liste de produits offerts et de voir leurs ventes. Pour les clients, ils ont la possibilité de voir leurs achats et leur solde de compte. Ce site web est disponible à tout moment.
