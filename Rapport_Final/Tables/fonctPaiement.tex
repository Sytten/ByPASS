\begin{table}[hp]
	\centering
	\caption{Fonctionnalités du terminal de paiement}
	\begin{tabular}{lP{2.5cm}P{2.5cm}P{2.5cm}P{2.5cm}ll}
	\hline
	\bf Code & \bf Fonctionnalité & \bf Objectif & \bf Description & \bf Contraintes & \bf Priorité & \bf Implanté \\
	\hline
	\hline
	2.1 &
	Effectuer un paiement &
	Payer un article  &
	Utiliser l’argent du compte client pour payer un produit &
	1 paiement par 2-3 secondes. Argent canadien. &
	1 &
	\checkmark \\\\
	%
	2.2 &
	Fonctionner sur batterie &
	Le terminal de paiement doit être portatif &
	Le terminal peut être facilement déplacé &
	Durée : environ 12h &
	1 &
	\checkmark\\\\
	%
	2.3 &
	Entrer le numéron de l’article acheté et la quantité &
	Définir un prix pour chaque article dans la BD&
	L’utilisateur doit être en mesure d’écrire le la quantité achetée de chaque article &
	Entre 1 et 9 produits par marchand &
	2 &
	\checkmark \\\\
	%
	2.4 &
	Afficher l’état du paiement &
	Connaître l’état du paiement &
	L’utilisateur doit être assuré que le paiement à fonctionner &
	Au moins les informations sur autoriser ou non &
	2 &
	\checkmark\\\\
	%
	2.5 &
	Lire le numéro d’un puce NFC &
	Identifier le client &
	Associer le numéro de carte à un compte dans la base de données &
	Doit être capable de lire la carte à 1 cm &
	3 &
	\checkmark\\\\
	%
	2.6 &
	Afficher le prix total de la transaction &
	Connaître le prix total du panier &
	L’utilisateur doit être en mesure de savoir le prix payé pour son achat &
	Montrer jusqu’au 5~sous. Max 999\$ &
	3 &
	\checkmark\\
	%
	2.7 &
	Détecter la position du terminal &
	Connaître l’emplacement du terminal &
	Être en mesure de connaître l’emplacement du terminal à tout moment &
	6-7 mètres près &
	4  \\\
	\end{tabular}
	\label{cahierPai}
\end{table}
