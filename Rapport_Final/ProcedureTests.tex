\section{Procédure de test du système du prototype}
	\subsection{Code C}
	Conséquemment aux ressources limitées du microcontrôleur, les tests unitaires et fonctionnels sont effectués lors d’une recompilation avec un \og flag\fg{} de test d'activité. Lorsque ce \og flag\fg{} est activé, une macro active des parties de code qui impriment des informations de déverminage à l’écran.

	Il est facile avec les informations de déverminage de détecter les valeurs erronées. Cette technique permet de cerner rapidement les sections de code problématiques lorsqu’un bug est détecté.

	\subsection{Code Javascript}
	Du côté du serveur javascript un \og framework\fg{} de test automatisé a été utilisé. L’équipe a utilisé une technique de développement TDD conforme aux règles de l’art de l’industrie. Le développement se faisait de la façon suivante : 
	%
	\begin{enumerate}
		\item Écriture d’un test d’intégration d’une fonctionnalité
		\item Écriture de la fonctionnalité
		\item Amélioration cyclique en continu
	\end{enumerate}%
	
	L’avantage de ce type de développement est qu’il permet à chaque développeur d’être agnostique du travail des autres. En effet, chaque développeur peut exécuter la commande \og npm test \fg{} pour que la suite de tests s’effectue. Si tous les tests passent, cela indique qu’ils n’ont pas brisé une fonctionnalité et qu’ils peuvent envoyer leur code sur github.

	La configuration du \og framework\fg{} de tests et des environnements de tests a été très long. Cependant, maintenant que c’est configuré, la productivité a nettement augmenté. L’équipe a donc été gagnante d’utiliser le TDD pour développer le projet.
		
	\subsection{Assurance qualité}
	\begin{longtable}[c]{P{0.3\textwidth}P{0.3\textwidth}P{0.3\textwidth}}
		\caption{Plan de test du prototype} \\
		\hline
		\bf Test & \bf Procédure & \bf Résultat attendu \\
		\hline
		\hline
		\csvreader[
			head to column names,
			late after line=\\\\,
			respect leftbrace = false,
			respect rightbrace = false]{Tables/QA_UTF8.csv}{}%
		{\Test & \Procedure & \ResultatAttendu}
%		\\\hline
	\end{longtable}
